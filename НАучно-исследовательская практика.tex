\input{top.tex}

\renewcommand{\bibname}{СПИСОК ИСПОЛЬЗОВАННЫХ ИСТОЧНИКОВ}
\renewcommand\refname{СПИСОК ИСПОЛЬЗОВАННЫХ ИСТОЧНИКОВ}

%\input{title.tex}

%\newpage
\setcounter{page}{2}
\thispagestyle {empty}
\renewcommand{\contentsname}{\centering СОДЕРЖАНИЕ}
\tableofcontents

\newpage
\section*{ВВЕДЕНИЕ}
\addcontentsline{toc}{section}{ВВЕДЕНИЕ}
Существуют разлиные подходы к изменению звукоотражающих характеристик тел в определенных направлениях. Изменение характеристик рассеяния звука упругих тел можно осуществить с помощью специальных покрытий. Представляет интерес исследовать звукоотражающие свойства тел с покрытиями в виде непрерывно неоднородного упругого слоя. Такой слой легко реализовать с помощью системы тонких однородных упругих слоев с различными значениями механических параметров (плотности и упругих постоянных).

В настоящей работе решается задача о рассеянии плоской монохраматической звуковой волны, падающей наклонно на упругий круговой цилиндр с неконцентрической полостью, покрытый радиально-неоднородным упругим слоем.

\newpage
\section{Постановка задачи} Рассмотрим бесконечный однородный упругий цилиндр с внешним радиусом $R_2,$ материал которого характеризуется плотностью $r_2$ и упругими постоянными $\lambda_2$ и $\mu_2.$ Цилиндр имеет произвольно расположенную цилиндрическую полость с радиусом $R_1.$ Оси цилиндра и полости являются параллельными. Цилиндр имеет покрытие в виде неоднородного изотропного упругого слоя, внешний радиус которого равен $R_3.$ Для решения задачи ввдем цилиндрические системы координат $r_2, \f_2, z_2$ и $r_1, \f_1, z_1,$ связанные с цилиндром и его полостью соответственно.

Полагаем, что модули упругости $\lambda_3$ и $\mu_3$ материала неоднородного цилиндрического слоя опиcываются дифференциируемыми функциями цилиндрической радиальной координаты $r,$ а плотность $\r$ -- непрерывной функцией координаты~$r.$ 

Будем считать, что окружающая цилиндр и находящаяся в его полости жидкость являются идеальными и однородными, имеющими в невозмущенном состоянии плотности $\r_1, \r_2$ и скорости звука $c_1, c_2$ соответственно.

Пусть из внешнего пространства на цилиндр произвольным образом падает плоская звуковая волна, потенциал скоростей которой равен
$$\P_0=A_0 \exp\{i[(\bar{k}\cdot \bar{r})-\omega t]\},$$
где $A_0$ -- амплитуда волны; $\bar{k}$ -- волновой вектор падающей волны; $\bar{r}$ -- радиус-вектор; $\omega$ -- круговая частота. В дальнейшем временной множитель $\exp\{-i\omega t\}$ будем опускать.

В цидиндрической системе координат падающая волна запишется в виде
$$\P_0=A_0\exp\{ik[r\sin\theta_0\cos(\f-\f_0)+z\cos\Q_0]\},$$
где $\theta_0$ и $\f_0$ -- полярный и азимутальный углы падения волны; $k=\omega / c$ -- волновое число во внешней области.

Определим отраженную от цилиндра волну и возбужденную в его полости звуковые волны, а также найдем поля смещений в упругом цилиндре и неоднородном слое.

\newpage
\section{Определение волновых полей}

Потенциал скоростей падающей плоской волны представим в виде
$$\P_0(r_1, \f_1, z_1)=A_0\exp\{i\alpha_1 z_1\}\sum_{n=-\infty}^{\infty}i^nJ_n(\beta_1 r_1)\exp\{in(\f_1-\f_{1_0})\},$$
где $J_n(x)$ -- цилиндрическая функция Бесселя порядка $n;$ $\alpha_1=k_1\cos\theta_0;$ \\$\beta_1=k_1\sin\theta_0.$

В установившемся режиме колебаний задача определения акустических полей вне цилиндра и внутри его полости заключается в нахожждении решений уравнения Гельмгольца
\begin{align}
&\Lap\P_1+k_1^2\P_1=0,\\
&\Lap\P_2+k_2^2\P_2=0\label{eq_gel_for_enviroment},
\end{align}
где $\P_1$ -- потенциал скоростей акустического поля в полости цилиндра;\\ $k_1=\frac{\omega}{c_1}$ -- волновое число жидкочти в полости цилиндра;
$\P_{2}$ -- потенциал скоростей полного акустического поля во внешней среде. 

В силу линейной постановки задачи
\begin{equation}
\P_2=\P_i+\P_s,
\end{equation}
где $\P_s$ -- потенциал скоростей рассеянной звуковой волны.

Тогда из \eqref{eq_gel_for_enviroment} получаем уравнение для нахождения $\P_s:$
\begin{equation}\label{eq__gel_for_psi_s}
\Lap\P_s+k_s^2\P_s=0.
\end{equation}

Уравнения \eqref{eq__gel_for_psi_s} и \eqref{eq_gel_for_enviroment} запишем в цилиндрических системах координат $r_1, \f_1, z_1$ и $r_2, \f_2, z_2$ соответственно. 

Отраженная волна $\P_s$ должна удовлетворять условиям излучения на бесконечности, а звуковая волна в полости цилиндра $\P_1$ -- условию ограниченности.

Поэтому потенциалы $\P_s$ и $\P_1$ будем искать в виде
\begin{align}
&\P_s(r_2, \f_2, z_2)= \exp\{i\alpha_2 z_2\}\sum_{n=-\infty}^{\infty}A_nH_n(\beta_2 r_2)\exp\{in(\f_2-\f_{2_0})\},\\
&\P_1(r_1, \f_1, z_1)= \exp\{i\alpha_1 z_1\}\sum_{n=-\infty}^{\infty}B_nH_n(\beta_1 r_1)\exp\{in(\f_1-\f_{1_0})\},
\end{align}
где $H_n(x)$ -- цилиндрическая функция Ханкеля первого рода порядка $n.$

Скорости частиц жидкости и акустические давления вне $(j=2)$ и внутри $(j=1)$ цилиндра определяются по следующим формулам соответственно:
\begin{equation}
\bar{\nu}_j=\grad\P_i;\:\:p_j=i\r_j\omega\P_j\:\:\:(j=1,2).
\end{equation}

Распространение малых возмущений в упругом теле для установившегося режима движения частиц тела описывается скалярным и векторным уравнениями Гельмгольца:
\begin{align}
&\Lap\P+k_l^2\P=0,\\
&\Lap\bar{\F}+k_{\tau}^2\bar{\F}=0,\label{fi_gel}
\end{align}
где $k_l=\omega/c_l$ и $k_{\tau}=\omega/c_{\tau}$ -- волновые числа продольных и поперечных упругих волн соответственно; $\P$ и $\bar\F$ -- скалярный и векторный потенциалы смещения соответственно; $c_l=\sqrt{(\lambda_1+2\mu_1)/\r_2}$ и $c_{\tau}=\sqrt{\mu_1/\r_2}$ -- скорости продольных и поперечных волн соответственно.

При этом вектор смещения $\bar{u}$ представляется в виде:
\begin{equation}
\bar{u}=\grad\P+\rot\bar{\F}.
\end{equation}

Векторное уравнение \eqref{fi_gel} в цилиндрической системе координат в обшем случае не распадается на три независимых скалярных уравнения относительно проекций вектора $\bar{\F},$ а представляет собой систему трех уравнений, решение которой сопряжено со значительными математическими трудностями.

Представим вектор $\F$ в виде
$$\F=\rot(L\bar{e}_z)+\frac1{k_\tau}\rot\rot(M\bar{e}_z)=\rot(L\bar{e}_z)+k_\tau M\bar{e}_z+\frac1{k_\tau}\grad\biggl(\frac{\partial M}{\partial z}\biggr),$$
где $L$ и $M$ -- скалярные функции пространственных координат $r, \phi, z;$ $\bar{e}_z$ -- единичный вектор оси z.

Тогда векторное уравнение \eqref{fi_gel} заменится двумя скалярными уравнениями Гельмгольца относительно функций $L$ и $M$
\begin{align*}
&\Lap L+k_{\tau}^2L=0,\\
&\Lap M+k_{\tau}^2M=0.
\end{align*}

С учетом условия ограниченности функции $\P, L$ и $M$ будем искать в виде
\begin{align}
&\P(r, \f, z)= \exp\{i\alpha z\}\sum_{n=-\infty}^{\infty}C_nH_n(k_1 r)\exp\{in(\f-\f_0)\},\\
&L(r, \f, z)= \exp\{i\alpha z\}\sum_{n=-\infty}^{\infty}D_nH_n(k_2 r)\exp\{in(\f-\f_0)\},\\
&M(r, \f, z)= \exp\{i\alpha z\}\sum_{n=-\infty}^{\infty}E_nH_n(k_2 r)\exp\{in(\f-\f_0)\},
\end{align}
где $k_1=\sqrt{k_l^2-\alpha^2}, k_2=\sqrt{k_{\tau}^2-\alpha^2}.$

Компоненты вектора смещения $\bar{u},$ записанные через функции $\P, L$ и $M$ в цилиндрической системе координат, имеют вид
\begin{align}
&u_r=\frac{\partial \P}{\partial r}+\frac{\partial^2 L}{\partial r\partial z}+\frac{k_\tau} {r}\frac{\partial M}{\partial \f},\\
&u_\f=\frac{1}{r}\frac{\partial \P}{\partial \f}+\frac{1}{r}\frac{\partial^{2} L}{\partial\f\: \partial z}-k_{\tau}\frac{\partial M}{\partial r},\\
&u_z=\frac{\partial\P}{\partial z}-\frac{\partial^{2} L}{\partial r^{2}}-\frac{1}{r}\frac{\partial L}{\partial r}-\frac{1}{r^2}\frac{\partial^{2}L}{\partial\f^{2}}.
\end{align}

Соотношения между компонентами тензора напряжений $\sigma_{ij}$ и вектора смещения $\bar{u}$ в однородном изотропном упругом цилиндре записываются следующим образом:
\begin{align}
&\sigma_{rr}=\lambda\Biggl(\frac{\partial u_r}{\partial r}+\frac{1}{r}\biggl(\frac{\partial u_\f}{\partial\f}+u_r\biggr)+\frac{\partial u_z}{\partial z}\Biggr)+2\:\mu\:\frac{\partial u_r}{\partial r},\\
&\sigma_{\f\f}=\lambda\Biggl(\frac{\partial u_r}{\partial r}+\frac{1}{r}\biggl(\frac{\partial u_\f}{\partial\f}+u_r\biggr)+\frac{\partial u_z}{\partial z}\Biggr)+2\:\mu\:\biggl(\frac{1}{r}\frac{\partial u_\f}{\partial\f}+\frac{u_r}{r}\biggr),\\
&\sigma_{zz}=\lambda\Biggl(\frac{\partial u_r}{\partial r}+\frac{1}{r}\biggl(\frac{\partial u_\f}{\partial\f}+u_r\biggr)+\frac{\partial u_z}{\partial z}\Biggr)+2\:\mu\:\frac{\partial u_z}{\partial z},\\
&\sigma_{r\f}=\mu\Biggl(\frac{1}{r}\frac{\partial u_r}{\partial\f}+\frac{\partial u_\f}{\partial r}-\frac{u_{\f}}{r}\Biggr),\\
&\sigma_{rz}=\mu\Biggl(\frac{\partial u_z}{\partial r}+\frac{\partial u_r}{\partial z}\Biggr),\\
&\sigma_{\f z}=\mu\Biggl(\frac{\partial u_\f}{\partial z}+\frac{1}{r}\frac{\partial u_z}{\partial \f}\Biggr).
\end{align}

Уравнения движения неоднородного изотропного упругого цилиндрического слоя в случае установившихся колебаний в цилиндрической системе координат имеют вид:
\begin{align}
\frac{\partial\sigma_{rr}}{
\partial r}+\frac{1}{r}\frac{\partial\sigma_{r\f}}{\partial\f}+\frac{\partial\sigma_{rz}}{\partial z}+\frac{\sigma_{rr}-\sigma_{\f\f}}{r}&=-\omega^2\rho(r)u_r,\\
\frac{\partial\sigma_{r\f}}{
\partial r}+\frac{1}{r}\frac{\partial\sigma_{\f\f}}{\partial\f}+\frac{\partial\sigma_{\f z}}{\partial z}+\frac{2}{r}\sigma_{r\f}&=-\omega^2\rho(r)u_{\f},\\
\frac{\partial\sigma_{rz}}{
\partial r}+\frac{1}{r}\frac{\partial\sigma_{\f z}}{\partial\f}+\frac{\partial\sigma_{zz}}{\partial z}+\frac{1}{r}\sigma_{rz}&=-\omega^2\rho(r)u_z.
\end{align}

\newpage
\section*{ЗАКЛЮЧЕНИЕ}
\addcontentsline{toc}{section}{ЗАКЛЮЧЕНИЕ}
Была поставлена задача о~дифракции плоских звуковых волн на~упругой сфере, имеющей произвольно расположенную полость и неоднородное покрытие. В~данной работе приведены основные уравнения колебаний, а также разложения в~ряд искомых функций для~внешней среды сферы, а также полости тела.

\end{document}